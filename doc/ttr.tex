% -*- latex -*-
\documentclass[11pt]{article}
\usepackage[backend=biber,citestyle=authoryear,style=alphabetic]{biblatex}

% \bibliography{PaperTools/bibtex/jp.bib}
\usepackage{graphicx}
\usepackage{grffile}
\usepackage{longtable}
\usepackage{wrapfig}
\usepackage{rotating}
\usepackage[normalem]{ulem}
\usepackage{amsmath}
\usepackage{amsthm}
\usepackage{textcomp}
\usepackage{amssymb}
\usepackage{capt-of}
\usepackage[dvipsnames]{xcolor}
\usepackage{hyperref}
\hypersetup{
    colorlinks,
    linkcolor={red!50!black},
    citecolor={blue!50!black},
    urlcolor={blue!80!black}
  }
\usepackage{mathpartir}
\usepackage{fontspec}
\usepackage{unicode-math}
\usepackage[plain]{fancyref}
\def\frefsecname{Section}
\def\freffigname{Figure}
\usepackage[colorinlistoftodos,prependcaption,textsize=tiny]{todonotes}
\usepackage{xargs}
\newcommandx{\unsure}[2][1=]{\todo[linecolor=red,backgroundcolor=red!25,bordercolor=red,#1]{#2}}
\newcommandx{\info}[2][1=]{\todo[linecolor=OliveGreen,backgroundcolor=OliveGreen!25,bordercolor=OliveGreen,#1]{#2}}
\newcommandx{\change}[2][1=]{\todo[linecolor=blue,backgroundcolor=blue!25,bordercolor=blue,#1]{#2}}
\newcommandx{\inconsistent}[2][1=]{\todo[linecolor=blue,backgroundcolor=blue!25,bordercolor=red,#1]{#2}}
\newcommandx{\improvement}[2][1=]{\todo[linecolor=Plum,backgroundcolor=Plum!25,bordercolor=Plum,#1]{#2}}
\newcommandx{\resolved}[2][1=]{\todo[linecolor=OliveGreen,backgroundcolor=OliveGreen!25,bordercolor=OliveGreen,#1]{#2}} % use this to mark a resolved question
\newcommandx{\thiswillnotshow}[2][1=]{\todo[disable,#1]{#2}} % will replace \resolved in the final document
% Link in bibliography interpreted as hyperlinks.
\newcommand{\HREF}[2]{\href{#1}{#2}}

\newtheorem{definition}{Definition}
\newtheorem{lemma}{Lemma}

\author{Jean-Philippe Bernardy}
\date{\today}
\title{Draft: a proof-theoretic presentation of TTR.}
\newcommand\ru[1]{\text{\sc #1}}
\begin{document}

\begin{mathpar}
  \ru{→-elim}
  \inferrule
  {  Γ ⊢ t : (x:A) → B
    \\Γ ⊢ u : A}{Γ ⊢ t u : B[u/x] }

  \ru{→-intro}
  \inferrule
  {Γ, x:A ⊢ t : B }
  {Γ ⊢ λx.t : (x:A) → B }

  \ru{rec-form-l}
  \inferrule
  {Γ, x:A ⊢ [\vec f] }
  {Γ ⊢ [x : A; \vec f ]  }

  \ru{rec-form-r}
  \inferrule
  { }
  {Γ ⊢ []  }

  \ru{rec-intro-l}
  \inferrule
  { }
  {Γ ⊢ (\vec f) : [ ]  }

  \ru{rec-intro-r}
  \inferrule
  {Γ ⊢ (\vec r) : [\vec f[t/x] ] \\ f=t ∈ \vec r \\ Γ ⊢ t : A  }
  {Γ ⊢ (\vec r) : [x : A; \vec f ]  }

  \ru{rec-elim}
  \inferrule
  {Γ ⊢ t : [\vec f]  \\ Γ ⊢ (t:[\vec f]).x : A}
  {Γ ⊢ t.x : A  }

  \ru{rec-elim-l}
  \inferrule
  { }
  {Γ ⊢ (t:[x:A; \vec f]).x : A  }

  \ru{rec-elim-r}
  \inferrule
  {Γ ⊢ (t:[\vec f [t.y/y]]).x : A   }
  {Γ ⊢ (t:[y:B; \vec f]).x : A  }

  \ru{conv}
  \inferrule
  {Γ ⊢ t : A  \\ A ⊂ B  }
  {Γ ⊢ t : B  }


  \ru{sub-rec-skip}
  \inferrule
    {[\vec f] ⊂ [\vec f']}
    {[x:A;\vec f] ⊂ [\vec f']}

  \ru{sub-rec-cong}
  \inferrule
    {[\vec f] ⊂ [\vec f'] \\ A ⊂ A'}
    {[x:A;\vec f] ⊂ [x:A';\vec f']}

  \ru{sub-→}
  \inferrule
    { A' ⊂ A\\B ⊂ B'}
    {(x:A) → B ⊂ (x:A') → B'}

  \ru{sub-refl}
  \inferrule
    { }
    {A ⊂ A}
\end{mathpar}
alternative subtyping rule for records: merge of A and A' equals A.
% \printbibliography
\end{document}
