% -*- latex -*-
\documentclass[11pt]{article}
\usepackage[backend=biber,citestyle=authoryear,style=alphabetic]{biblatex}

% \bibliography{PaperTools/bibtex/jp.bib}
\usepackage{graphicx}
\usepackage{grffile}
\usepackage{longtable}
\usepackage{wrapfig}
\usepackage{rotating}
\usepackage[normalem]{ulem}
\usepackage{amsmath}
\usepackage{amsthm}
\usepackage{textcomp}
\usepackage{amssymb}
\usepackage{capt-of}
\usepackage[dvipsnames]{xcolor}
\usepackage{hyperref}
\hypersetup{
    colorlinks,
    linkcolor={red!50!black},
    citecolor={blue!50!black},
    urlcolor={blue!80!black}
  }
\usepackage{mathpartir}
\usepackage{fontspec}
\usepackage{unicode-math}
\usepackage[margin=1.5cm]{geometry}
\usepackage[plain]{fancyref}
\def\frefsecname{Section}
\def\freffigname{Figure}
\usepackage[colorinlistoftodos,prependcaption,textsize=tiny]{todonotes}
\usepackage{xargs}
\newcommandx{\unsure}[2][1=]{\todo[linecolor=red,backgroundcolor=red!25,bordercolor=red,#1]{#2}}
\newcommandx{\info}[2][1=]{\todo[linecolor=OliveGreen,backgroundcolor=OliveGreen!25,bordercolor=OliveGreen,#1]{#2}}
\newcommandx{\change}[2][1=]{\todo[linecolor=blue,backgroundcolor=blue!25,bordercolor=blue,#1]{#2}}
\newcommandx{\inconsistent}[2][1=]{\todo[linecolor=blue,backgroundcolor=blue!25,bordercolor=red,#1]{#2}}
\newcommandx{\improvement}[2][1=]{\todo[linecolor=Plum,backgroundcolor=Plum!25,bordercolor=Plum,#1]{#2}}
\newcommandx{\resolved}[2][1=]{\todo[linecolor=OliveGreen,backgroundcolor=OliveGreen!25,bordercolor=OliveGreen,#1]{#2}} % use this to mark a resolved question
\newcommandx{\thiswillnotshow}[2][1=]{\todo[disable,#1]{#2}} % will replace \resolved in the final document
% Link in bibliography interpreted as hyperlinks.
\newcommand{\HREF}[2]{\href{#1}{#2}}
\newcommand{\istype}{: \Type}

\newtheorem{definition}{Definition}
\newtheorem{lemma}{Lemma}
\newtheorem{theorem}{Theorem}
\author{Jean-Philippe Bernardy}
\date{\today}
\title{TTR formal system}
\newcommand\ru[1]{\text{\sc #1}}
\newcommand\Type{\mathsf{Type}}
\newcommand\splt[1]{\mathsf{split} \{ #1 \}}
\newcommand\fin[1]{\{ #1 \}}
\newcommand\conv{⟷}
\newcommand\eval[2]{⟦#1⟧_{#2}}
\newcommand\app[2]{\mathbf{app}(#1,#2)}
\newcommand\proj[2]{#1 • #2}
\begin{document}

This note describes a syntax, type checking rules and type-checking
algorithm for an intuitionistic type-theory. The goal is to capture
Cooper's TTR.
\subsection*{Syntax}
The syntax for type and proof terms is as follow. Note in particular
that, as customary for Russel-style dependent-types, the syntaxes of types and proof
terms are unified.
\begin{align*}
  t,u,A,B & ::=  x \\
          & ∣~    λx. t \\
          & ∣~    t u & \text{application} \\
          & ∣~    (l_i = t_i) & \text{record} \\
          & ∣~    t.l & \text{projection} \\
          & ∣~    `l & \text{tag} \\
          & ∣~    \splt {`l_i ↦ t_i, …} \\
          & ∣~    (t : A) & \text{ annotated term }\\
          & ∣~    \Type \\
          % & ∣~    ⊥ \\
          & ∣~    (x:A) → B \\
          & ∣~    [ l_i : A_i ] \\
          & ∣~    \fin{ `l_i } \\
          & ∣~    A ∧ B ~|~ A ∨ B 
\end{align*}
\begin{align*}
  Γ,Δ & ::=  — \\
      & ∣  x:A, Γ
\end{align*}

Example types:
\begin{align*}
  ⊥ & = \fin {} \\
  Σ(x:A)(B[x]) & = [fst : A ; snd : B[fst]] \\
  A × B & = Σ(\_:A)B \\
  A → B & = (\_:A) → B \\
  A + B & = [choice : \fin {`a, `b}; payload : \splt {`a ↦ A; `b ↦ B}]
\end{align*}

\subsection*{Type checking: $Γ ⊢ t : A$}
The type checking relation is defined inductively by the following rules:
\begin{mathpar}
  \ru{type-form}
  \inferrule
  { }
  {Γ ⊢ \Type \istype}

  % \ru{$⊥$-form}
  % \inferrule
  % { }
  % {Γ ⊢ ⊥  \istype}

  \ru{$→$-form}
  \inferrule
  {Γ ⊢ A \istype \\ Γ, x:A ⊢ B \istype }
  {Γ ⊢ (x:A) → B \istype}

  \ru{rec-form-l}
  \inferrule
  {Γ ⊢ A \istype \\ Γ, x:A ⊢ [l_1 : A_1[x] … l_n : A_n[x]] \istype }
  {Γ ⊢ [l : A; l_1 : A_1[l] … l_n : A_n[l] ] \istype }

  \ru{rec-form-r}
  \inferrule
  { }
  {Γ ⊢ []  \istype }

  \ru{fin-form}
  \inferrule
  { }
  {Γ ⊢ \{`l_0 … `l_n \} \istype }

  \ru{meet-form}
  \inferrule
  {Γ ⊢ A \istype \\ Γ ⊢ B \istype }
  {Γ ⊢ A ∧ B \istype}

  \ru{join-form}
  \inferrule
  {Γ ⊢ A \istype \\ Γ ⊢ B \istype }
  {Γ ⊢ A ∨ B \istype}

\end{mathpar}

\begin{mathpar}

  \ru{var}
  \inferrule
  {x : A ∈ Γ }
  {Γ ⊢ x : A }

  \ru{$→$-elim}
  \inferrule
  {  Γ ⊢ t : (x:A) → B
    \\Γ ⊢ u : A}{Γ ⊢ t u : B[u/x] }

  \ru{$→$-intro}
  \inferrule
  {Γ, x:A ⊢ t : B }
  {Γ ⊢ λx.t : (x:A) → B }

  \ru{rec-intro-l}
  \inferrule
  { }
  {Γ ⊢ () : [ ]  }

  \ru{rec-intro-r}
  \inferrule
  {Γ ⊢ t_0 : A_0 \\ Γ ⊢ (k_0=t_0 …  k_m=t_m) : [l_1:A_1[t/l] … l_n:A_n[t/l] ]   }
  {Γ ⊢ (l_0=t_0, k_1=t_1 … k_m=t_m) : [l_0:A_0 ; l_1:A_1 … l_n:A_n  ]  }

  \ru{rec-elim}
  \inferrule
  {Γ ⊢ t : [l_0 : A_0; …; l_{n+1} : A_{n+1} ] }
  {Γ ⊢ t.l_{n+1} : A_{n+1}[t.l_0/l_0 … t.l_n/l_n]   }

  \ru{fin-intro}
  \inferrule
  {  }
  {Γ ⊢~  `l : \fin { `l }  }

  \ru{fin-elim}
  \inferrule
  {Γ ⊢ t_i : A[l_i]  }
  {Γ ⊢ \splt {`l_0 ↦ t_0, …`l_n ↦ t_n} : (x : \fin { `l_i, … }) → A[x] }

  % \ru{meet-intro}
  % \inferrule
  % {Γ ⊢ t : A  \\ Γ ⊢ t : B }
  % {Γ ⊢ t : A ∧ B }

  % \ru{join-intro}
  % \inferrule
  % {Γ ⊢ t : A   }
  % {Γ ⊢ t : A ∨ B }

  \ru{sub}
  \inferrule
  {Γ ⊢ t : A \\ \eval A Γ ⊑ \eval B Γ  }
  {Γ ⊢ t : B }

\end{mathpar}

Remark: there are no rules for meet and join because the conversion
rules takes care of those types. See in particular the subtyping
relation below.

In the \ru{sub} rule we write $\eval A Γ$ to mean
$\eval A {[x_i↦X_i, ∀ x_i:C_i ∈ Γ]}$ (see evaluation below) --- that
is, every variable is substituted for an unknown value.


\subsection*{Values}
Neutral values:
\begin{align*}
  n,v,w,V,W & ::=  X & \text{unknown value} \\
      & ∣~    ⟨x,t,ρ⟩ & \text{closure (suspended function evaluation)} \\
      & ∣~    n v & \text{application} \\
      & ∣~    (l_i = t_i) & \text{record} \\
      & ∣~    n.l & \text{projection} \\
      & ∣~    `l & \text{tag} \\
      & ∣~    \splt {`l_i ↦ v_i, …} \\
      & ∣~    \Type \\
      % & ∣~    ⊥ \\
      & ∣~    Π V W \\
      & ∣~    [] \\
      & ∣~    [l : V ; W ] \\
      & ∣~    \fin{ `l_i }  & \text{finite set} \\
      & ∣~    V ∧ W ~|~ V ∨ W 
\end{align*}
Environments map variables to values:
\begin{align*}
  ρ ::= [] ∣ [ρ, x ↦ V]
\end{align*}

\subsection*{Evaluation: $\eval t ρ$}
The evaluation function maps a term $t$ and an environment $ρ$ to a
value $\eval t ρ$. It assumes that all free variables in $t$ are bound
in $ρ$.
\begin{align*}
  \eval {x} ρ &= ρ(x) \\
  \eval {λx.t} ρ &= ⟨ x,t,ρ⟩ \\
  \eval {t u} ρ &= \app {\eval t ρ} {\eval u ρ} \\
  \eval {(l_0 = t_0, … l_n = t_n)} ρ &= (l_0 = \eval{t_0} ρ, \eval{(l_i=t_i[x/l_0])} {ρ,x↦\eval{t_0}ρ}) \\
  \eval {t.l} ρ &= \proj {\eval t ρ} l   \\
  \eval {\splt{x_i↦t_i}} ρ &=  {\splt{x_i↦\eval{t_i}ρ}}\\
  \eval {\Type} ρ &= \Type \\
  \eval {(x:A) → B} ρ &= Π \eval A ρ \eval {λx.B}ρ \\
  \eval {[l_0 : A_0;…;l_n = A_n]} ρ &= l_0 : \eval {A_0} ρ; \eval{λx.[l_i : A_i[x/l_0]]}ρ  \\
  \eval {[]} ρ &= [] \\
  \eval {\fin{`l_i}} ρ &= \fin{`l_i} \\
  \eval {A ∧ B} ρ &= \eval A ρ ∧ \eval B ρ  \\
  \eval {A ∨ B} ρ &= \eval A ρ ∨ \eval B ρ
\end{align*}

\begin{align*}
 \app{⟨x,t,ρ⟩} u & = \eval t {ρ,x↦u} \\
 \app{\splt{…;l↦t;…}} {`l} & = t \\
 \app n u & = n u & \text{otherwise} \\
 \proj {(…,l=t,…)} l & = t \\
 \proj n l & = n.l & \text{otherwise}
\end{align*}
\subsection*{Subtyping: $V ⊑ W$}
Subtyping is a relation on values defined as follows:
\begin{mathpar}

  \ru{sub-top}
  \inferrule
    { }
    {V ⊑ []} % [] has no elimination rule

  \ru{sub-rec-skip}
  \inferrule
    {W ⊑ \app {W'} X}
    {[l_0:V_0; W] ⊑ W'}

  \ru{sub-rec-cong}
  \inferrule
    {V_0 ⊑ V'_0 \\ \app W X  ⊑ \app {W'} X}
    {[l_0:V_0; W] ⊑ [k_0:V'_0; W']}

  \ru{sub-$→$}
  \inferrule
    {V' ⊑ V\\ \app W X ⊑ \app {W'} X}
    {Π V W ⊑ Π V' W'}

  \ru{sub-$∧$-l}
  \inferrule
    {V ⊑ V'}
    {V ∧ W ⊑ V'}

  \ru{sub-$∧$-r}
  \inferrule
    {V ⊑ V' \\ V ⊑ W' }
    {V ⊑ V' ∧ W'}

  \ru{sub-$∨$-l}
  \inferrule
    {V ⊑ V' \\ W ⊑ V' }
    {V ∨ W ⊑ V'  }

  \ru{sub-$∨$-r}
  \inferrule
    {V ⊑ V' }
    {V ⊑ V' ∨ W'  }

  \ru{sub-fin}
  \inferrule
    {\fin {`l_0 … `l_n} ⊆ \fin {`k_0 … `k_n} }
    {\fin {`l_0 … `l_n} ⊑ \fin {`k_0 … `k_n} }

  \ru{sub-refl}
  \inferrule
    {V \conv W}
    {V ⊑ W}

  \ru{sub-bot}
  \inferrule
    { }
    {⊥ ⊑ V} % ⊥ has no introduction rule
  \end{mathpar}


  % The rule for application would be:
  % \begin{mathpar}
  % \inferrule
  %   {t ⊑ t' \\ u⊑u' \\ u'⊑u} 
  %   {t u ⊑ t' u' \\ (*)}
  % \end{mathpar}

  % because there can be both positive and negative occurences of the
  % argument. (consider $t = t' = λx. x → x$). And so would be other
  % eliminators. So in the end it's better to have only the reflection
  % rule and drop the rules for eliminators and introductions of
  % non-type rules.

  $⊑$ forms a lattice with
  $∧$ as meet,
  $∨$ as join,
  $⊥$ as least element and
  $[]$ as greatest element.
\subsection*{Convertibility: $t \conv u$}
The convertibility relation is the smallest transitive
reflexive congruence containing the following rules:

\begin{mathpar}
  \inferrule
  {V \conv U \\ W \conv U}
  {V ∨ W \conv U}

  \inferrule
  {V \conv U \\ W \conv U}
  {V ∧ W \conv U}

  \inferrule
  {⟨ x,t,ρ⟩ \conv v \\ X ~\text{fresh}}
  {\eval t {ρ,x↦X} \conv \app v X}
\end{mathpar}
(symmetric rules are also included.)

Note: because of subtyping, a complete congruence may not be
required. That is we need to check that the terms are
compatible. Thus, two records will be considered convertible if the
matching fields are convertible. Likewise for splits.

\newcommand\chk{:\!\!?\hspace{2pt}}
\newcommand\ifr{:!}
\newcommand{\ifrtype}{\ifr \Type}

\subsection*{Algorithmic rules: $Γ ⊢ t \chk V$ and $Γ ⊢ t \ifr V$}
The judgement $Γ ⊢ t : A$ is unfortunately not decidable, because it
requires conjuring up types for certain intermediate results. Thus we
define another judgement which has a decidable form, but demands
certain types to be explicitly written.

Checking rules:
\begin{mathpar}
  \ru{check-rec-intro-l}
  \inferrule
  { }
  {Γ ⊢ (l_i ↦ t_i) \chk [ ]  }

  \ru{check-rec-intro-r}
  \inferrule
  {Γ ⊢ t \chk V  \\ Γ ⊢ (l_i = t_i) \chk [\app W {\eval t Γ} ] \\ (l=t) ∈ \fin {l_i = t_i} }
  {Γ ⊢ (l_i = t_i) \chk [l : V; W]  }

  \ru{check-pi-intro}
  \inferrule
  {Γ, x:V ⊢ t \chk \app W X }
  {Γ ⊢ λx.t \chk Π V W }

  \ru{check-conv}
  \inferrule
  {Γ ⊢ t \ifr V \\ V ⊑ W }
  {Γ ⊢ t \chk W }

 \ru{check-fin-elim}
 \inferrule
  {Γ ⊢ t_i \chk \app {V} {l_i}  }
  {Γ ⊢ \splt {`l_i ↦ t_i, …} \chk Π \fin { `l_i } V}

\end{mathpar}

Inference rules:
\begin{mathpar}
  \ru{inf-var}
  \inferrule
  {x : V ∈ Γ }
  {Γ ⊢ x \ifr V }

  \ru{inf-ann}
  \inferrule
  {Γ ⊢ A \chk \Type \\ Γ ⊢ t \chk \eval A Γ}
  {Γ ⊢ (t : A) \ifr \eval A Γ }

  \ru{inf-fin-intro}
  \inferrule
  { }
  {Γ ⊢ `x \ifr \fin {`x} }

  \ru{inf-pi-elim}
  \inferrule
  {Γ ⊢ t \ifr Π V W \\ Γ ⊢ u \chk V }
  {Γ ⊢ t u \ifr \app W {\eval u Γ} }

 \ru{inf-rec-elim}
 \inferrule
 {Γ ⊢ t \ifr [l_0 : V_0 ; W]  \\ Γ ⊢ (t:[l_0 : V_0 ; W]).l_k \ifr A}
 {Γ ⊢ t.l_k \ifr A  }

 \ru{inf-rec-elim-l}
 \inferrule
 { }
 {Γ ⊢ (t:[l:V; W]).l \ifr V  }

 \ru{inf-rec-elim-r}
 \inferrule
 {Γ ⊢ (t:\app W {\proj {\eval t Γ} {l_0}} \ifr A   }
 {Γ ⊢ (t:[l_0:V_0; W]).l \ifr W  }

 \ru{inf-type-form}
 \inferrule
 { }
 {Γ ⊢ \Type \ifrtype}
 
  % \ru{$⊥$-form}
  % \inferrule
  % { }
  % {Γ ⊢ ⊥  \ifrtype}

  \ru{inf-$→$-form}
  \inferrule
  {Γ ⊢ A \ifrtype \\ Γ, x:A ⊢ B \ifrtype }
  {Γ ⊢ (x:A) → B \ifrtype}

  \ru{inf-rec-form-l}
  \inferrule
  {Γ ⊢ A \ifrtype \\ Γ, x:A ⊢ [l_1 : A_1[x] … l_n : A_n[x]] \ifrtype }
  {Γ ⊢ [l : A; l_1 : A_1[l] … l_n : A_n[l] ] \ifrtype }

  \ru{inf-rec-form-r}
  \inferrule
  { }
  {Γ ⊢ []  \ifrtype }

  \ru{inf-fin-form}
  \inferrule
  { }
  {Γ ⊢ \{`l_0 … `l_n \} \ifrtype }

  \ru{inf-meet-form}
  \inferrule
  {Γ ⊢ A \ifrtype \\ Γ ⊢ B \ifrtype }
  {Γ ⊢ A ∧ B \ifrtype}

  \ru{inf-join-form}
  \inferrule
  {Γ ⊢ A \ifrtype \\ Γ ⊢ B \ifrtype }
  {Γ ⊢ A ∨ B \ifrtype}
 
\end{mathpar}

\begin{theorem}
  If $Γ ⊢ t \chk V$, then $\eval t ρ$ terminates.
\end{theorem}
\begin{proof}
  We first observe that
  \begin{itemize}
  \item The type is already evaluated
  \item The term has a correct type
  \end{itemize}
  This allows to build a sound model for our theory into a language of
  total functions. In the interest of economy, we will map it to
  another type-theory, say Agda. Then:
  \begin{itemize}
  \item Interpreting values into Agda is obviously terminating, as no
    reduction can take place. We can call this function El : Type ->
    Set. (Where Set is an Agda Set). This function can be extended to
    contexts in a usual way, and called Env.

    Because the syntax and the semantics are so close, the
    interpretation is straightforward. The noteworthy bit is that
    union types are interpreted to disjoint unions, and intersection
    types as simple products.
  \item If $V ⊑ W$, then we can construct a function $El V → El W$.
  \item We can then define evaluation in Agda as a follows:
    $(ρ : Env Γ) → (t : Term) → Γ ⊢ t \chk V → V$. Erasing the last
    argument yields our evaluation function.
  \end{itemize}

\end{proof}

\begin{theorem}
Convertibility and subtyping are decidable relations.
\end{theorem}


\begin{theorem}
  The relations $Γ ⊢ t \chk V$ and  $Γ ⊢ t \ifr V$  are decidable.
\end{theorem}
\begin{proof}
  The relations are structural over the proof terms. (Except for the
  record checking rule; but then the record types gets smaller in the
  local induction.)
\end{proof}

\subsection*{Elaboration}
The decidable system requires the user to add type annotations at
certain introduction rules. If one adds those annotations using the
system below, one gets a term whose type is decidable.
\newcommand\trp[1]{\left(#1\right)^+}
\newcommand\trn[1]{\left(#1\right)^-}

\begin{definition}[Elaboration of proofs]
  The proofs are translated as follows. For concision we show only the
  proof terms, but the translation requires the whole typing
  derivation (and thus that the term typing is decided already).
  \begin{align*}
    \trp{x} & = x \\
    \trp{t u} & = \trp t \trn u \\
    \trp{t.l} & = \trn t.l \\
    \trp{`l} & = `l \\
    \trp{u} & = (\trn u : T)  & \text {otherwise --- note that $T$ comes from the typing derivation} \\
    \trn{λx.t} & = (λx. \trn t) \\
    \trn{(l_0 = t_0, … , l_n = t_n )} & = {(l_0 = \trn t_0, … , l_n = \trn {t_n} )} \\
    \trn{\splt {`l_i ↦ t_i, …} } & = {\splt {`l_i ↦ \trn {t_i}, …} } \\
    \trn{t} & = \trp{t} & \text {otherwise} \\
  \end{align*}
\end{definition}

The idea is to translate terms whose type are inferable using the +
translation, leaving the others to the - translation. If + is used on
a non-inferrable term, then an annotation is added.

\begin{theorem}~
  \begin{itemize}
  \item if $Γ ⊢ t : A$ then $Γ ⊢ \trn t \chk \eval A Γ$.
  \item if $Γ ⊢ t : A$ then $Γ ⊢ \trp t \ifr W$ and $W ⊑ \eval A Γ$.
  \end{itemize}
\end{theorem}
\begin{proof}
  By induction on the typing derivation.
\end{proof}

\begin{theorem}
  if $Γ ⊢ t \chk \eval A Γ$ then $Γ ⊢  t : A$.
\end{theorem}
\begin{proof}
  By erasing the type annotations in the proof term.
\end{proof}

\subsection*{Acks}
This system is based on
``A simple type-theoretic language: Mini-TT'',
Coquand et al.
\url{http://www.cse.chalmers.se/~bengt/papers/GKminiTT.pdf}

\subsection*{Convenience}
In this section we add some rules that are but can be useful to add to
avoid type annotations.

Typing:
\begin{mathpar}
  \ru{check-rec-elim}
  \inferrule
  {Γ ⊢ t \chk [l : V]  }
  {Γ ⊢ t.l \chk V  }

 \ru{inf-fin-elim}
 \inferrule
  {Γ ⊢ t_i \ifr V_i  }
  {Γ ⊢ \splt {`l_i ↦ t_i, …} \ifr Π \fin { `l_i } ⋀_i V_i }
\end{mathpar}

Simplification equations for meet and join (to use during evaluation):
\begin{align*}
  [] ∧ V       &= V \\
  ⊥ ∧ V       &= ⊥ \\
  [l:V;W] ∧ [l:V';W']  &= [l:(V ∧ V');(\app W X ∧ \app {W'} X)] \\
  [l:V;W] ∧ [W']       &= [l:V;\app W X ∧ W']   \\
  \{ l_i \} ∧ \{ l'_i \} & = \{ l_i \} ∩ \{ l'_i \} \\
  V ∧ V & = V \\
  V ∧ W & = ⊥ & \text {otherwise and if neither $A$ or $B$ is neutral.} \\
\end{align*}
(symmetrically for join)
% (Π V W) ∧ (Π V' W') &= Π (V ∨ V') (ΛX. \app W X ∧ \app {W'} X) \\ % nonsense

% \begin{align*}
%   ((x:A) → B) ∨ ((x:A') → B') &= (x:A ∧ A') → (B ∨ B') \\
%   [\vec f] ∨ [\vec f']                  &= [\vec f ∨ \vec f'] \\
%   ─ ∨ \vec f       &= ─ \\
%   (l:A;\vec f) ∨ \vec f'       &= (\vec f ∨ \vec f')  & x ∉ \vec f' \\
%   (l:A;\vec f) ∨ (l:A';\vec f')       &= l:(A ∨ A');(\vec f ∨ \vec f') \\
%   \{ \vec t \} ∨ \{ \vec t' \} & = \{ \vec t ∪ \vec t' \} \\
%   A ∨ A & = A \\
%   A ∨ B & = [] & \text {otherwise and if neither $A$ or $B$ is neutral.}
% \end{align*}

% Perhaps have a parallel execution as the introductor of meet?
% (t|u) : A ∧ B
% (t|u) v = (t v|u v)
% (t|u).x = (t.x|u.x)
% (λx.t|λx.u) = λx.(t|u)
% (t|crash) = t


% \printbibliography
\end{document}
